\documentclass[a4paper]{article}

\usepackage{physics}

\title{Solutions to exercises from Ramamurti Shankar "Quantum Mechanics"}
\date{}

\begin{document}

\maketitle
\tableofcontents

\section{Chapter 1}

\paragraph{Exercise 1.1.1}

\begin{enumerate}
    \item There is exactly one null vector.

    Let $\ket{0}, \ket{0'}$ be null vectors against addition operation. Then for any $\ket{v}$ we have
    $$
        \ket{v} + \ket{0} = \ket{0} + \ket{v} = \ket{0},
    $$
    $$
        \ket{v} + \ket{0'} = \ket{0'} + \ket{v} = \ket{0'}.
    $$
    For that reason $\ket{0} + \ket{0'} = \ket{0}$. Similarly $\ket{0} + \ket{0'} = \ket{0'}$. Eventually, $\ket{0} = \ket{0'}$.

    \item $0 \ket{v} = \ket{0}$.

    We have $\ket{0} = \ket{v} + \ket{-v} = (0 + 1) \ket{v} + \ket{-v} = 0 \ket{v} + \ket{v} + \ket{-v} = 0 \ket{v} + \ket{0} = \ket{0}$.

    \item $\ket{- v} = - \ket{v}$.

    We have $\ket{v} + \ket{-v} = \ket{0} = \textrm{(point 2)} = 0 \ket{v} = (1 + (-1)) \ket{v} = \ket{v} + (- \ket{v})$

    For that reason $\ket{- v} = - \ket{v}$

    \item for any $\ket{v}$ there is exactly one $\ket{- v}$.

    For any $\ket{v}$ let $\ket{w}$ it's inversion. There is only one null vector $\ket{0}$ (according to point $1$). For that reason $\ket{v} + \ket{w} = \ket{v} + (- \ket{v})$. We have then $\ket{w} = - \ket{v}$, because there is exactly one inversion. 
\end{enumerate}

\paragraph{Exercise 1.1.2}

$$
    (a, b, c) + (d, e, f) = (a + d, b + e, c + f)
$$
$$
    \alpha (a, b, c) = (\alpha a, \alpha b, \alpha c)
$$
$$
    \textrm{null vector:} \ket{0} = (0, 0, 0)
$$
$$
    \textrm{inverse vector:} (-a, -b, -c)
$$
Vectors $(a, b, 1)$ don't create linear space, because their set don't have null vector.

\paragraph{Exercise 1.1.3}

\begin{enumerate}
    \item Yes
    \item No, because there is not null vector
    \item No, because there is not null vector
\end{enumerate}

\paragraph{Exercise 1.1.4}

Space of these vectors isn't linearly independent, because

$$
    -1 \ket{1} + 2 \ket{2} + 1 \ket{3} = \ket{0}.
$$

\paragraph{Exercise 1.1.5}

\begin{enumerate}
    \item This set of vectors isn't linearly independent, because we have $\ket{1} = \frac{1}{2} (\ket{3} - \ket{2})$
    \item This set of vectors is linearly independent, because only solution of $a_{1} \ket{1} + a_{2} \ket{2} + a_{3} \ket{3} = \ket{0}$ is $a_{1} = a_{2} = a_{3} = 0$:

    $$
        a_{1} (1, 1, 0) + a_{2} (1, 0, 1) + a_{3} (0, 1, 1) = (0, 0, 0)
    $$

    $$
        \begin{cases}
            a_{1} + a_{2} = 0 \\
            a_{1} + a_{3} = 0 \\
            a_{2} + a_{3} = 0
        \end{cases}
    $$
\end{enumerate}

\end{document}
